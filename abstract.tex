\begin{abstract}
\textbf{Context}: Code smells or bad smells are an accepted approach to identify design flaws in the source code. Although it has been widely explored by researchers, it is subject to programmers interpretation. One way to handle this subjectivity is to use Machine Learning techniques.

\noindent\textbf{Objective}: This paper provides the reader an overview of Machine Learning techniques and code smells found in literature, aiming to identify which methods and practices are mostly used when applying machine learning for code smells identification and which machine learning techniques are being applied for code smells identification

\noindent\textbf{Method}: A mapping study was used to identify the techniques used for each smell.

\noindent\textbf{Results}: We found that the Bloaters was the main kind of smell studied,  addressed by 35\% of the articles. The most commonly used technique was Genetic Algorithms, used by 22.22\% of the articles. Regarding the smells addressed by which technique, there was a high level of redundancy, in a way that the smells are covered by a wide range of algorithms. Nevertheless, Feature Envy stood out, being aimed by 63\% of the techniques. When it comes to performance, the best average  was provided by Decision Tree, followed by Random Forest, Semi-supervised and Support Vector Machine Classifier techniques.

\noindent\textbf{Conclusions}: From the 25 analyzed smells, only 5 were not approached by any Machine Learning techniques. Most of them focus in several code smells and in general there is no outperforming technique, except for few specific smells. We also found a lack of comparable results due to the heterogeneity of the data sources and of the provided results. We recommend further empirical studies to assess the performance of these techniques in a standardized data-set to improve the comparison reliability and reproducibility.
\end{abstract}